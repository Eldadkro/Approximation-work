\documentclass[]{article}



\usepackage[utf8]{inputenc}
\usepackage[T1]{fontenc}
\usepackage{mathptmx}
\usepackage[english]{babel}
\usepackage{graphicx}
\usepackage{amsmath}
\usepackage{amssymb}
\usepackage{amsfonts}
\usepackage{amsthm}
\usepackage{subcaption}

\usepackage{biblatex}
\addbibresource{./bibli.bib}

\usepackage{hyperref}
\graphicspath{ {plots/} }
\hypersetup{
	linktocpage=true,
	colorlinks=true,
	linkcolor=red,
	filecolor=magenta,      
	urlcolor=cyan,
	pdftitle={Approximation Work},
	pdfpagemode=FullScreen,
	bookmarks=true,
	breaklinks=true
}

\urlstyle{same}
\newtheorem{thm}{Theorem}[section]
\newtheorem{cor}[thm]{Corollary}
\newtheorem{lem}[thm]{Lemma}
\newtheorem{definition}{Definition}[]

\newcommand{\norm}[1]{\left\lVert#1\right\rVert}
\newcommand{\Q}[0]{\textit{Q}}

\DeclareMathOperator*{\argmax}{argmax}
\DeclareMathOperator*{\argmin}{argmin}


%opening
\title{Approximation Final Work}
\author{Eldad Kronfeld}
\date{}

\begin{document}

\maketitle
\tableofcontents

\newpage
\section{Theoretical Question }
\subsection{Background}
\begin{definition}
	\label{definition1}
	Let $f$ be a continuous function on the interval $[a, b]$, and let $p$ be a polynomial approximation. An alternating set on $f, p$ is defined to be a sequence of points $x_{0}, ..., x_{n-1}$ such that:
	\begin{itemize}
	\item $a \leq x_0 <...<x_{n - 1}\leq b .$
	\item $f(x_i) - p(x_i) = (-1)^i E \text{ for } i=0,1,...,n-1.$ 
\end{itemize}
where either $E=||f-p||$ or $E=-||f-p||$. The number $n$ is the length of the alternating set.
\end{definition}


\begin{thm}
	\label{thm1}
	Let $f \in C[a.b]$, and suppose that $p=p^{*}_{n}$ is the best approximation of $f$ out of $P_{n}[x]$, that is
	\begin{equation}
		||f-p||_{\infty} \le ||f-q||_{\infty} \text{ for all } q\in P_{n}[x] 
	\end{equation}
	Then, there is an alternating set for $f-p$ consisting of at least $n+2$  points.
	
\end{thm}
\subsection{Question}
prove that the best approximation polynomial is unique using Theorem (\ref{thm1}). 
\begin{proof}
	The theorem is trivial if $f$ is polynomial of degree $\leq n$, so we assume that is not true. to show the uniqueness, suppose that both $p_n, q_n$ are polynomials of best approximation, and we will show that they are equal.
	Note that $\frac{(p_n + q_n)}{2}$ is also a polynomial of best approximation because:
	\begin{equation*}
		 \norm{f-\frac{(p_n + q_n)}{2}} = \norm{\frac{(f - p_n)}{2} \frac{(f - q_n)}{2}} \leq \frac{1}{2}\norm{f-p_n} + 
		 \frac{1}{2}\norm{f-q_n} = \pm E
	\end{equation*}
	therefore, there are $n+2$ points at which:
	\begin{equation*}
		\frac{f-p_n}{2}+\frac{f-q_n}{2}=\pm E
	\end{equation*}
	At each of those alternating points both $f-p_n$ and $f-q_n$ are equal to $\pm E$ then we can say that at each of the $n+2$ points:
	\begin{equation*}
		(f(x_i) - p_n(x_i)) - (f(x_i)-q_n(x_i)) = 0 \text{ for each of the alternating point, }i=1..n+2
	\end{equation*}
	Since both $p_n,q_n$ are polynomials of degree$\leq$n then they must be identical, therefore $p_n$ is unique.
\end{proof}

\section{Practical question}
\subsection{Background}
\begin{definition}
	the \textbf{Chebyshev center}\cite{convexOptimization} of a bounded set $\textit{Q}\subseteq X$ with non-empty interior is the center of the smallest ball that encloses the entire of the set $\Q$,and the radius is:
	\begin{equation*}
		r=\inf\{\sup\{\norm{x-y}:x\in\Q\}:y\in X\}
	\end{equation*}
\end{definition}
The Chebyshev center is described as the solution to the following optimization problem:
\begin{equation}
	\label{OP1}
	\begin{aligned}
		\min_{\hat{x},r} \quad &r \\
		\text{s.t.}\quad &\norm{\hat{x} - x} \leq r \text{ for all }x\in\Q \\
		\quad & r\geq 0
	\end{aligned}
\end{equation}
The optimization problem states that the radius r is the target function we would like to minimize is subjected to two types of constraints:
\begin{itemize}
	\item All the points of \Q are inside.
	\item $r$ needs to be non-negative, as it is the radius.
\end{itemize}
Those it could also be described as the solution to the min-max problem.
\begin{equation}
	\label{OP2}
	\argmin_{\hat{x}} \max_{x\in\Q} \space \norm{x-\hat{x}}
\end{equation}
which translate to the minimization of the distance of the furthest point from the center. Both of the optimization problems correspond to the first definition of Chebyshev centers.\\
\textbf{Note:} other books and papers\cite{ParametricOptimization} define the Chebyshev centers differently which is not equivalent to the definition (\ref{definition1}).
\begin{definition}
	\label{definition2}
	The \textbf{Chebyshev center} of a bounded set $\Q\subseteq X$ with non-empty interior is the center of the largest ball inscribed in $\Q$, and the center is:
	\begin{equation}	
		\hat{x}=\argmax_{x\in X}\space d(x,\Q) = \argmax_{x\in X} \space \min_{y\in \Q} \norm{\hat{x} - x}
	\end{equation}
\end{definition}
The difference between the two definitions (\ref{definition1}) and (\ref{definition2}) is that definition (\ref{definition1}) denote the center that encloses the set $\Q$ like in figure (???). However the second definition (\ref{definition2}) states that the ball should be enclosed by the set $\Q$, like in figure (???) and not the other way around.

\paragraph{Chebyshev center as a Linear programming problem \\}
In contrast to the optimization problems above (\ref{OP1}, \ref{OP2}) the definition used in order to contrast the linear problem is (\ref{definition2}). In order to define the problem the set $\Q$ must be represented as an intersection of finitely many half-spaces. The intersection, if non-empty, creates a convex polytope, which is a geometrical object with flat faces created by the intersection of half-spaces as can be seen in figure \ref{Fig:polytope}.

\begin{figure}[!h]
	
	\begin{subfigure}{0.5\textwidth}
		\includegraphics[width=1\linewidth,height=\textwidth]{Tetrahedron}
		\caption{tetrahedron}
	\end{subfigure}
	\begin{subfigure}{0.5\textwidth}
		\includegraphics[width=1\linewidth,height=\textwidth]{3dpoly}
		\caption{3-dimensional convex polytope}
	\end{subfigure}
	\caption{Two examples for convex polytopes.}
	\label{Fig:polytope}
\end{figure}

\printbibliography[
heading=bibintoc,
title={References}
]

\end{document}
